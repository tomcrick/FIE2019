\font\manual=manfnt
\def\dbend{{\manual\char127}}
\def\eqq{{\buildrel?\over=}}
\def\C{\mathbf{ C}}
\def\N{\mathbf{ N}}
\def\Q{\mathbf{ Q}}
\def\R{\mathbf{ R}}
\def\Z{\mathbf{Z}}
\def\F{\mathbf{F}}
\def\softO{\tilde{\cal O}}
\def\O{{\cal O}}
%\def\N{{\bf N}}
%\def\Q{{\bf Q}}
%\def\R{{\bf R}}
%\def\Z{{\bf Z}}
\def\action#1{\hfil\rlap{\bf#1}\break}
\def\pcite#1{[#1]}
\def\Res{\mathop{\rm Res}\nolimits}
\def\r#1#2{``#1''$\rightarrow$``#2''}
\documentclass{article}
\bibliographystyle{alphaurl}
\usepackage[hyphens]{url}
%\usepackage{enumitem}
\usepackage{pdfpages}
\usepackage{verbatim}
\usepackage{hyperref}
\usepackage[show]{ed}
\usepackage{graphicx}
\newtheorem{proposition}{Proposition}
\newtheorem{theorem}{Theorem}
\newtheorem{lemma}{Lemma}
\newtheorem{corollary}{Corollary}
\newtheorem{definition}{Definition}
\newtheorem{notation}{Notation}
\newtheorem{example}{Example}
\newtheorem{problem}{Problem}
\def\decision#1{\par{\bf #1}}
\pagestyle{empty}
\begin{document}
\author{Crick/Davenport/Irons/Prickett}
\title{Response to Referees}
\date{June 2019}
\maketitle

\section{Referee 1}
No comments that require addressing in detail.
\section{Referee 2}
No comments that require addressing in detail.
\section{Referee 3}
\begin{enumerate}
\item ``The research appears to be poorly structured and the analysis/argument is hard to interpret''.
\item[A1]We have done what we can, but note that the other referees disagree. We have attempted to make the structure clearer.
\item ``The abstract is too long and could be made concise. For example, the 2nd paragraph can be avoided as the main idea has been conveyed in the 1st paragraph as well.''
\item[A]We have adjusted the second paragraph of the abstract to provide an indication of the research design adopted and the rationale for doing so. Cybersecurity is now a commonly included curricula element within the delivery of general computer science. Evaluating how cybersecurity is taught in all aspects of computer science hence is considerably large task. We feel that is beyond the scope of our current research and the scope of this paper. We have instead adopted a case study based research approach to evaluate the evidence of the teaching of cybersecurity within general computer science. We have adjusted the abstract to indicate this and attempted to more clearly indicate the case study based approach adopted in the paper.
\item What do the authors mean by ``There has been a recent international working group but this has yet to report.'' 
\item[A] In parallel to our research, their is a working group as part of the Association for Computing Machinery (ACM) Conference on Innovation and Technology in Computer Science Education intending to develop a taxonomy of cybersecurity education. We have clarified this in text.
\item It would add more clarity to introduce what these terms DevOps vs DevSecOps represent in the paper. Also, please use the acronym GDPR after introducing it during the first usage such as General Data Protection Regulation (GDPR). 
\item[A] Discussion related to DevOps vs DevSecOps has been reduced. We apologize for the manner in which we have employed abbreviations and have adopted the style recommended above.
\item There are repeated ``and'' in the sentence ``In particular the expectation of Privacy by design...'' Also, the ``by'' after ``and'' is not required. 

Please consider rephrasing to ``While both Privacy by design and privacy by defualt have been expected to be good practices''.

\item[A] These have been rephrased, apologies for our weak use of the English language in this section. One of the authors is dyslexic and we should have better proof read some of the sections.
\item Please rephrase the following statement ``How might accreditation regard cybersecurity education, or help with it?'' as the sentence structure doesn't make much sense. 
\item[A] We have rephrased to "Can accreditation enhance the provision of cybersecurity within an accrediting bodies jurisdiction?"
\item It is not clear what is the message the authors are conveying in the 2nd paragraph? It lists three different web links and a reference. Also, the web links in 2nd and 3rd paragraphs could be moved to the reference and they could be referenced with a number in this paragraph. 
Again, please use the acronym KA after introducing it as Knowledge Areas (KA) at the first instance of use. 


\item[A]
\item  What is JHD? Please expand?
\item[A] Sorry we should have removed this. 
\item Please expand OWASP. 
\item[A]We have done so. Also SQL.
\item 
The references to first author in the 2nd paragraph could be avoided and the sentence rephrased to highlight only the facts and not any personal experience. 
\item[A] We have removed the references to the authors throughout
\item 
 What does ``three accounted for the 36 of the 44'' mean? Please clarify. I am assuming the 36 and 44 refer to number of universities. Please add that clarification, if it is so.
\item[A] We have clarified, we can confirm it did mean universities.
\item 
``If there are only 82 instances of these fragments, how can 117 of them be verified?'' Please add clarification to this discussion.
\item[A]
\end{enumerate}
\section{Referee 4}
\begin{enumerate}
\item ``The problem is it is not a research paper in even the broadest interpretation of that sense. The RQs are invented based on topic areas the authors wish to riff on, only RQ3 comes (somewhat) close to a RQ with an associated presentation based on the literature and some analysis. The 1st 2 RQs, as well as the Introduction sections, are selectively argued, stream-of-consciousness almost writing. Entertaining, informative, arguable, debatable - but not research. No methodology is applied, no systematic review, etc. Instead a smattering of anecdotal information and quotes (and even more quotes in the footnotes) intended to convince the reader of a viewpoint that is never quite articulated clearly - until the end when it comes down to a small set of specific complaints about SQL Injection, XSS, and StackOverflow (which is overall a bit of a letdown as I felt as if this bloglike writing was building to a more impactful crescendo, some kind of kick-in-the-pants call to action for the community).''
\item[A]
\end{enumerate}
\section{Programme Committee}
No comments that require addressing in detail.

\end{document}
\item 
\item[A]
\item 
\item[A]
\item 
\item[A]


\end{document}
\bibliography{../../../../jhd}
\section{}
\begin{description}
\item
\end{description}
\begin{itemize}
\item
\end{itemize}
\begin{example}
\end{example}
\begin{definition}
\end{definition}
\begin{theorem}
\end{theorem}
\begin{enumerate}
\end{enumerate}
\begin{description}
\item[Theme 1]
\item[Theme 2]
\item[Theme 3]
\item[Theme 4]
\item[Theme 5]
\end{description}

