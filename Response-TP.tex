\font\manual=manfnt
\def\dbend{{\manual\char127}}
\def\eqq{{\buildrel?\over=}}
\def\C{\mathbf{ C}}
\def\N{\mathbf{ N}}
\def\Q{\mathbf{ Q}}
\def\R{\mathbf{ R}}
\def\Z{\mathbf{Z}}
\def\F{\mathbf{F}}
\def\softO{\tilde{\cal O}}
\def\O{{\cal O}}
%\def\N{{\bf N}}
%\def\Q{{\bf Q}}
%\def\R{{\bf R}}
%\def\Z{{\bf Z}}
\def\action#1{\hfil\rlap{\bf#1}\break}
\def\pcite#1{[#1]}
\def\Res{\mathop{\rm Res}\nolimits}
\def\r#1#2{``#1''$\rightarrow$``#2''}
\documentclass{article}
\bibliographystyle{alphaurl}
\usepackage[hyphens]{url}
%\usepackage{enumitem}
\usepackage{pdfpages}
\usepackage{verbatim}
\usepackage{hyperref}
\usepackage[show]{ed}
\usepackage{graphicx}
\newtheorem{proposition}{Proposition}
\newtheorem{theorem}{Theorem}
\newtheorem{lemma}{Lemma}
\newtheorem{corollary}{Corollary}
\newtheorem{definition}{Definition}
\newtheorem{notation}{Notation}
\newtheorem{example}{Example}
\newtheorem{problem}{Problem}
\def\decision#1{\par{\bf #1}}
\pagestyle{empty}
\begin{document}
\author{Crick/Davenport/Irons/Prickett}
\title{Response to Referees}
\date{June 2019}
\maketitle

\section{Referee 1}
No comments that require addressing in detail.
\section{Referee 2}
No comments that require addressing in detail.
\section{Referee 3}
\begin{enumerate}
\item ``The research appears to be poorly structured and the analysis/argument is hard to interpret''.
\item[A1]The paper has been re-written to  make the structure clearer. In particular, the abstract has been updated to more succinctly provide the essence of the work; increased the amount of signposting to guide the reader; added a research approach section to more clearly explain the approach followed; made the research questions more precise to communicate the intentions more clearly; and made the sections more self contained to better support the reading of the paper in a non-linear manner.
\item ``The abstract is too long and could be made concise. For example, the 2nd paragraph can be avoided as the main idea has been conveyed in the 1st paragraph as well.''
\item[A] The abstract is now two paragraphs long and has been enhanced to better present the essence of the work.
\item What do the authors mean by ``There has been a recent international working group but this has yet to report.'' 
\item[A] In parallel to the research explored in the paper, their is a working group as part of the Association for Computing Machinery (ACM) Conference on Innovation and Technology in Computer Science Education intending to develop a taxonomy of cybersecurity education. This has been clarified.
\item It would add more clarity to introduce what these terms DevOps vs DevSecOps represent in the paper. Also, please use the acronym GDPR after introducing it during the first usage such as General Data Protection Regulation (GDPR). 
\item[A] Discussion related to DevOps vs DevSecOps has been reduced. Abbreviations have adopted the style recommended. 
\item There are repeated ``and'' in the sentence ``In particular the expectation of Privacy by design...'' Also, the ``by'' after ``and'' is not required. 

Please consider rephrasing to ``While both Privacy by design and privacy by defualt have been expected to be good practices''.

\item[A] These sentences have been rephrased and other typographical errors addressed.

\item Please rephrase the following statement ``How might accreditation regard cybersecurity education, or help with it?'' as the sentence structure doesn't make much sense. 
\item[A] This has been rephrased to "Can accreditation by Professional, Statutory and Regulatory Bodies enhance the provision of cybersecurity within a bodies jurisdiction?" The other research questions have also been rephrased to improve the clarity by which the intentions of the research are communicated.
\item It is not clear what is the message the authors are conveying in the 2nd paragraph? It lists three different web links and a reference. Also, the web links in 2nd and 3rd paragraphs could be moved to the reference and they could be referenced with a number in this paragraph. 
Again, please use the acronym KA after introducing it as Knowledge Areas (KA) at the first instance of use. 


\item[A] The authors recognize these where note sections that should not have been present in the paper. These sections have been removed. The recommended approach to abbreviations has now been adopted throughout the paper.
\item  What is JHD? Please expand?
\item[A] This has been removed this. 
\item Please expand OWASP. 
\item[A]Open Web Application Security Project has replaced the initial use of OWASP.
\item 
The references to first author in the 2nd paragraph could be avoided and the sentence rephrased to highlight only the facts and not any personal experience. 
\item[A] The references to the authors has been removed throughout.
\item 
 What does ``three accounted for the 36 of the 44'' mean? Please clarify. I am assuming the 36 and 44 refer to number of universities. Please add that clarification, if it is so.
\item[A] This did mean universities and has been updated to indicate this.
\item 
``If there are only 82 instances of these fragments, how can 117 of them be verified?'' Please add clarification to this discussion.
\item[A]
\end{enumerate}
\section{Referee 4}
\begin{enumerate}
\item ``The problem is it is not a research paper in even the broadest interpretation of that sense. The RQs are invented based on topic areas the authors wish to riff on, only RQ3 comes (somewhat) close to a RQ with an associated presentation based on the literature and some analysis. The 1st 2 RQs, as well as the Introduction sections, are selectively argued, stream-of-consciousness almost writing. Entertaining, informative, arguable, debatable - but not research. No methodology is applied, no systematic review, etc. Instead a smattering of anecdotal information and quotes (and even more quotes in the footnotes) intended to convince the reader of a viewpoint that is never quite articulated clearly - until the end when it comes down to a small set of specific complaints about SQL Injection, XSS, and StackOverflow (which is overall a bit of a letdown as I felt as if this bloglike writing was building to a more impactful crescendo, some kind of kick-in-the-pants call to action for the community).''
\item[A] The paper has been revised to more clearly communicate the research approach employed. A research approach section has been added to further elicit the case study approach adopted. The research questions have been made more specific to more clearly communicate the intentions of the research. The writing style has been adjusted and a more formal approach taken. The conclusions have been extended and clarified.
\end{enumerate}
\section{Programme Committee}
No comments that require addressing in detail.

\end{document}
\item 
\item[A]
\item 
\item[A]
\item 
\item[A]


\end{document}
\bibliography{../../../../jhd}
\section{}
\begin{description}
\item
\end{description}
\begin{itemize}
\item
\end{itemize}
\begin{example}
\end{example}
\begin{definition}
\end{definition}
\begin{theorem}
\end{theorem}
\begin{enumerate}
\end{enumerate}
\begin{description}
\item[Theme 1]
\item[Theme 2]
\item[Theme 3]
\item[Theme 4]
\item[Theme 5]
\end{description}

