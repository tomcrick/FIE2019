\documentclass[conference]{IEEEtran}
\IEEEoverridecommandlockouts
% The preceding line is only needed to identify funding in the first footnote. If that is unneeded, please comment it out.
\usepackage{paralist}
\usepackage{url}
\usepackage{enumitem}
\usepackage[hyphenbreaks]{breakurl}
\usepackage[T1]{fontenc}
\usepackage{booktabs} % For formal tables

\usepackage{hyperref}

\def\UrlBreaks{\do\/\do-}

\begin{document}

\title{Cybersecurity Is Too Important\\To Be Left To The Specialists}
%\title{What Do We Mean By Cybersecurity Education?}  Tom's original

% \author{\IEEEauthorblockN{Tom Crick}
% \IEEEauthorblockA{\textit{School of Education} \\
% \textit{Swansea University}\\
% Swansea, UK \\
% thomas.crick@swansea.ac.uk}
% \and
% \IEEEauthorblockN{James H. Davenport}
% \IEEEauthorblockA{\textit{Department of Computer Science} \\
% \textit{University of Bath}\\
% Bath, UK \\
% j.h.davenport@bath.ac.uk}
% \and
% \IEEEauthorblockN{Alastair Irons}
% \IEEEauthorblockA{\textit{Faculty of Technology} \\
% \textit{name of organization (of Aff.)}\\
% Sunderland University, Sunderland, UK\\
% alastair.irons@sunderland.ac.uk}
% \and
% \IEEEauthorblockN{Tom Prickett}
% \IEEEauthorblockA{\textit{School of Computing} \\
% \textit{name of organization (of Aff.)}\\
% Northumbria University, Newcastle upon Tyne, UK \\
% tom.prickett@northumbria.ac.uk}
% }

% compact IEEE author format for more than three authors...
\author{\IEEEauthorblockN{Tom Crick\IEEEauthorrefmark{1}, James
    H. Davenport\IEEEauthorrefmark{2}, Alastair
    Irons\IEEEauthorrefmark{3} and Tom Prickett\IEEEauthorrefmark{4}} 
\IEEEauthorblockA{\IEEEauthorrefmark{1}School of Education, 
Swansea University, Swansea, UK; Email: thomas.crick@swansea.ac.uk} 
\IEEEauthorblockA{\IEEEauthorrefmark{2}Department of Computer Science, 
University of Bath, Bath, UK; Email: j.h.davenport@bath.ac.uk}
\IEEEauthorblockA{\IEEEauthorrefmark{3}Faculty of Technology, 
Sunderland University, Sunderland, UK; Email: alastair.irons@sunderland.ac.uk}
\IEEEauthorblockA{\IEEEauthorrefmark{4}School of Computing, 
Northumbria University, Newcastle upon Tyne, UK; Email: tom.prickett@northumbria.ac.uk}}

\maketitle

\def\IntroYear{2014}  % when BCS started asking for CyberSecurity

\begin{abstract}
% submitted abstract (100-500 word limit):
There are numerous facets to cybersecurity education, from theory to
practice, hardware and software, to social and technical (as well as
other important dimensions). A multitude of national and international
model curricula and recommendations - from national academies, learned
societies and even governments -- have been presented and discussed in
recent years, with varying levels of impact on policy and practice. In
this paper we attempt to address the key questions ``what should the
generalist computer scientist/engineer know'' and ``how well is that
being taught''. (Cyber)security -- if thought about at all -- has
historically been left to specialists, frequently viewed as a
masters-level discipline; or left for development in professional
practice, sometimes called ``information assurance'' in the UK and
Europe, which worked, even if not well, in a sequential process model.

For example, in the world of software and systems development, we have
seen the disruption of Agile (invented c.2001) and DevOps (invented
c.2009), but not necessarily through the lens of security; however,
more recently there has been a recognition of the need for change,
exemplified by the emergence of ``DevSecOps''. If security is not to
be left to the experts, then the generalist must know about it. Thus,
in the context of widespread international computer
science/engineering curriculum reform - both in compulsory, as well as
post-compulsory education - what does this trend mean more generally
for institutions and educators, and how do we teach it?

In this paper, we frame a key social, cultural and economic global
challenge of cybersecurity in computing and engineering education by
analysing the UK's national security, economic, and skills policy
context, shaping the practical considerations for education
professionals from schools through to universities (and into industry
and professional practice). Through this analysis of UK educational
policy and practice, we make a number of recommendations for future
cybersecurity educational initiatives and interventions, including a
number of innovative practical suggestions for educators and
curriculum designers, with a view for potential adoption and adaption
in other jurisdictions internationally.
\end{abstract}

\begin{IEEEkeywords}
TBC
\end{IEEEkeywords}


\section{Introduction}

Cybersecurity has been in the news for several years, generally prompted by spectacular breaches of one kind or another, such as \cite{BritishAirways2018a}.
It needs attention across the spectrum:
\begin{quote}
change the culture in your organisation around cyber security; to try to do for cyber what has been done so successfully for health and safety, for example, over the last ten years - to get everybody to take it seriously; to take the risk management process seriously and drive that down through the organisation. \cite{Hannigan2019a}\footnote{Former Director of GCHQ: British equivalent of NSA.}
\end{quote}

There are calls for education to respond to this situation, which it does both at the individual level and via recommended curricula \cite{ACM2013a} and professional accreditation requirements \cite{BCS2018a}. There has been a recent international working group \cite{Parrishetal2018a}, but this has yet to report.  We note that part of its third aim is to ``catalog existing [\dots] knowledge materials'', but there is no mention of any quality control over these (see section \ref{sec:SQL}).

Nevertheless, it is one thing to write curricula and requirements, and
another thing to deliver appropriate education, and one could
reasonably ask how well this is done in practice.
\subsection{CyberSecurity: for all or for specialists?}
In one sense, this title is a false dichotomy: there is a serious need for Cybersecurity specialists (estimates vary, but are always large: \cite{JCNSS2018a} has only anecdotal evidence), but also all in IT need to know \emph{some} Cybersecurity, as the recent fashion for talking about DevSecOps rather than just DevOps exemplifies.

This is not a brand-new concern: see \cite{Parr2014a} for concerns over five years ago, but it is a growing one. On the one hand, GDPR has increased the corporate penalties for failure, and therefore the demand for various forms of Cybersecurity specialists, both explicitly by requiring Data Protection Officers\footnote{\cite{EU292016a} places quite strong requirements on these: ``expertise in national and European data protection laws and practices and an in-depth understanding of the GDPR'', as well as ``principles of data processing'', ``data protection by design and by default'' and ``security of processing''. It is hard to envisage one person possessing all these attributes, and indeed ``DPO as a service'' is being promoted \cite{McCreanor2018d}.} and implicitly by causing boards to invest more in Cybersecurity. On the other hard, the growing publicity given to these, and the growing use of electronic payment, has also led to greater demand for staff in this area.
% JHD what are these?  Also the next paragraph needs thinking about

AI: GDPR essentially expects CyberSecurity to be in all coding. 
JHD: PCI DSS insists on a Web Application Firewall, which is only needed if you don't trust your applications.


Specialist curricula abound: a recent one is \cite{ACMIEEEAISSIGSECIFIP}. However, given the shortage of specialists, in practice a good generalist with some Cybersecurity expertise can often get a ``specialist'' job.
\subsection{Research Questions}
There are various levels of specialism at which Cybersecurity education can be addressed.
\begin{enumerate}
\item The person in the street --- this is important, but there are many initiatives in this area, which are, rightly, largely separated from computing education.
% ({\emph{interesting point from a public engagement/science communication perspective -- aren't ``we'' trying to create a digitally competent and capable citizenry? Minor point but maybe worth mentioning e.g. being safe online, accessing public services, etc}})
%\item[JHD]This is important, agreed, but I think this is being a user: think car driving rather than car design. BUT the point I heard from the WC Engineers ``how can security be an option --- safety certainly isn't'' is also relevant.
\item The general CS graduate.
\item The general CS masters graduate.
\item The specialist CS graduate.
\item The specialist CS masters graduate.
\end{enumerate}
The focus of this paper is on 2--3: the general CS graduate.
\begin{description}
\item[RQ1]What should be taught to the generalist, and how?
\item[RQ2]Should this be taught stand-alone or integrated? 
\item[RQ3]How might accreditation regard Cybersecurity Education, or help with it?
\end{description}
\cite[p.~97]{ACM2013a} takes a distinct view on RQ2:
\begin{quote}
The Information Assurance and Security KA is unique among the set of KAs presented here
given the manner in which the topics are pervasive throughout other Knowledge Areas.
\end{quote}
It proposes 9 ``core'' hours and 63.5 distributed across the other Knowledge Areas.
%\pagebreak - to fix URL issues
\par
Nevertheless, the situation on the ground is different: \cite{Ackerman2019a}, describing the USA, writes as follows
\begin{quote}
Universities suffer shortcomings, as well. Roughly 85 of them offer undergraduate and/or graduate degrees in cybersecurity. There is a big catch, however. Far more diversified computer science programs, which attract substantially more students, don't mandate even one cybersecurity course.
\end{quote}

But \cite[Table 1]{Ruiz2019a} shows that the UK situation is distinctly different: he quotes that 61\% of UK courses offer mandatory cybersecurity content, and his research was based on web scraping, it represents a lower bound.
\section{Policy Context}


% TC to add policy+skills context
% inc. new EU Cybersecurity Act: https://ec.europa.eu/commission/news/cybersecurity-act-2018-dec-11_en

There has been substantial CS/digital curriculum reform across the
UK~\cite{crick+sentance:2011,brown-et-al:sigcse2013,wgictreview:2013,brown-et-al:toce2014,moller+crick:jce2018}
--- but what has happened to the focus on Cybersecurity?  The Institute of Coding \cite{Davenportetal2019a}, designed to improve digital skills for UK graduates, does indeed mention Cybersecurity, but merely as a sub-item in one work package (1.2a).

UK national economic skills priority
e.g. \url{https://www.gov.uk/government/publications/national-cyber-security-strategy-2016-to-2021}
%(esp. \url{https://publications.parliament.uk/pa/jt201719/jtselect/jtnatsec/706/70605.htm}
(especially \cite{JCNSS2018a}
and \url{https://www.ncsc.gov.uk/blog/skills-and-training} and
\url{https://www.cybersecuritychallenge.org.uk/}). 

There is large-scale media
attention on the ``Cybersecurity skills gap''
e.g. \url{https://www.contracts.mod.uk/do-features-and-articles/digital-skills-shortage-threatens-uk-cyber-security/}
and
\url{https://www.itpro.co.uk/cyber-security/31554/uk-government-lacks-urgency-in-tackling-cyber-security-skills-gap}
etc -- link to UK Digital Strategy and UK Industrial Strategy.

\section{Challenges}

What should be in the general curriculum.  Pervasive (as in Maths or LSEPI) rather than specialist option.  
{\bf JHD to research ACM general curriculum;}

The UK's official knowledge resource, the CyBOK project \cite{Bristol2019a}, has produced reference documentation for some (2 final, 3 for comment out of a planned total of 19) knowledge areas, which are useful references for the experienced educator looking for a definition or characterisation, but a long way frm being a textbook (which is not their aim).

Is teaching Cybersecurity different? Lecturing is probably not the best way. Should we use real-life case studies (e.g. British Airways)

%AI noted the workshops with ISC$^2$, with the ``this is a fad'' concept.  {\bf AI to write up}.
  Should we be teaching it because it's underpinning? because there's a skill shortage?  JHD would argue both.

% should we frame this as one or more case studies?
% do we have any national survey data?

\subsection{PCI DSS}\label{sec:PCIDSS}
The Payments Card Industry Data Security Standards \cite{PCI2018b} underpin all processing of credit/debit cards. Nevertheless, they are very rarely mentioned in generalist computer scientist courses --- one payments industry person commented ``I've given up even asking if recruits have heard of PCI DSS: it's so rare''. 

This would not matter so much if everyone handling payments data wer sent by their employers on an effective\footnote{``Effective'' is important: the Processing Director at one major acquirer commented ``I just caught two developers in the elevator discussing how to meet a customer requirement, and I had to tell them this was contrary to PCI DSS. Their response was that the customer wanted it!''.} PCI DSS course. However, the payments business is now so spread across websites, often run by SMEs, or non-specialists, that some provision should be made in courses.  Even larger enterprises are not immune: \cite{Barth2018a} reports that the recent British Airways breach was caused by a failure to adhere to PCI DSS in the website maintenance.

%Alastair to write about their Credit Fraud experience (MSc Cybersecurity).  HE Academy grant (to be referenced).


\subsection{Educational Resources: SQL Injection}\label{sec:SQL}
It is 15 years since \cite{Guimaraesetal2004} 
wrote ``All the topics listed above should be presented in the first
Database Course'', and the first such topic was SQL injection \cite{SPIDynamics2002,Anonymous2018b}. SQL injection as an attack has been around for twenty years \cite{HornerHyslip2017a}, has its own cartoon (\url{https://xkcd.com/327/}, dating back to 2007 according to the Internet Archive) and website (\url{http://bobby-tables.com/}). Nevertheless SQL injection is still a major weakness: number one in the OWASP Top 10 \cite{OWASP2017a}, and has been in the Top 10 since at least 2003.  \cite[the UK's definitive reference]{Bristol2018a} states ``a wide range of attack techniques for exploiting SQL
injection or script injection are known and documented.''.

Clearly such a major weakness should be well-taught. One would like to think it is, but the first author has yet to meet a UK undergraduate who recalls being taught it \emph{outside} a specialist Cybersecurity course.  In general it's hard to determine what is taught, but a reasonable proxy for this is the content of recomended textbooks. 

Hence \cite{Drop2019} analyzed the database textbooks used by  44 of the top 50 Computer Science
departments in the United States (using \cite{StangerMartin2015a}'s list, the other six didn't have a book listed). There were seven such books, but three accounted for the 36 of the 44. Five of the seven (30 of the 44) had no mention of SQL injection. On the other two, the more popular one has a seriously flawed discussion\footnote{``However, they imply that using parameters is equivalent to using a function to add escape characters
around user input. This is incorrect, as using parameters allows
SQL statements to be pre-compiled, and prevents any user input
from being interpreted as code, while escaping user input is not
recommended as a sole defense since imperfect escape functions
can easily be subverted.'' \cite{Drop2019}}, and the other, while generally excellent, had a presentational problem\footnote{``However, the fact that the first
example should not be used is not discussed until two pages after
the example in the text, and is not mentioned at all in the caption or
on the page where the figure appears. This means a student who is
skimming the text looking for an example to modify for their own
code could simply copy the code that first appears in the example,
without being aware that this is in fact an example of what they
shouldn't do.'' \cite{Drop2019}}.
\par
This blindness is not limited to textbooks: although Wikipedia has an excellent article on SQL Injection, it was not linked from the SQL page itself.\footnote{The authors intend to fix this after the paper is processed, so as not to break anonymity.}
\subsection{Informal Resources}\label{sec:informal}
The web abounds with informal resources: tutorials and code snippets. How good are these, and how good are people at using these? This has been looked at by \cite{Unruhetal2017a}, who took the top five results from Google for six queries. Of these 30 tutorials, 6 had SQL injection weaknesses, and 3 had Cross-Site Scripting\footnote{Number 7 in OWASP's Top Ten \cite{OWASP2017a}.} weaknesses. Searching for these fragments in PHP projects on GitHub found 82 instances of these fragments, of which 117 were verified by manual to be vulnerable --- 80\% of which were vulnerable to SQL injection.

\subsection{The case of Java}\label{sec:Java}
To the best of the authors' knowledge, no survey equivalent to \cite{Drop2019} has been done for Java textbooks. Indeed, many such books go nowhere near security applications.  But this means that the programmer who has to implement security is left to the documentation of the package/API being used, and to informal resources. \cite{Mengetal2018a} analysed 503 Cybersecurity-related postings on the popular Stack overflow (\url{https://stackoverflow.com}) resource.  53\% were about the Spring Security framework (\url{https://projects.spring.io/spring-security/}), dominated by authentication (45\%). The discussion \cite[\S4.3.1]{Mengetal2018a} of cross-site request forgery (CSRF) is especially worrying.  By default, Spring implcitly enables protection against this. But all the accepted answers to CSRF-related failures simply suggested disabling they check. There were no negative comments about this, and indeed a typical response is ``Adding \verb!csrf().disable()!
solved the issue!!! I have no idea why it was enabled by default''. As of writing (16 January 2018) there were no negative comments about this disabling of a vital security feature.
\par
This research was further developed by \cite{Chenetal2019a}  (and popularised in a security community in \cite{Zorz2019a}). Their first finding was
\begin{quote}
644 out of the 1,429 inspected answer posts
(45\%) are insecure, meaning that insecure suggestions
popularly exist on SO. Insecure answers dominate, in
particular, the SSL/TLS category [355 insecure versus 150 secure]].
\end{quote} 

\subsection{Android}\label{sec:Android}
\cite{Fischeretal2017a} looked specifically at the use of resources from Stack Overflow in Android applications. Their key finding was this.
\begin{quote}
We found that 15.4\% of all 1.3 million Android applications
contained security-related code snippets from
Stack Overflow. Out of these 97.9\% contain at least one
insecure code snippet.
\end{quote}
We should note two caveats (in opposite directions). Their labelling was conservative, in that snippets were only labelled as insecure if that was demonstrable, and, for example, mere use of outdated SSL/TLS was not automatically deemed insecure. On the other hand, the insecure snippet might have been used in a way that did not expose the insecurity.
\par
The uncritical reading of Stack Overflow was also noted in \cite[Slide 29]{Votipkaetal2019a}. Their key recommendation  \cite[Slide 32]{Votipkaetal2019a} is ``Improve documentation: Clarify what you can(not) copy/paste''. 
\subsection{Staff}
It is well known that Cybersecurity skills are in short supply, for example \cite{Page2018a}.
\begin{quote}
% JHD: I am in touch with ESG to try to find more, and someUK-specific numbers.
Research into the state of IT conducted annually by ESG\footnote{Apparently \url{https://www.esg-global.com/research/esg-brief-2018-cybersecurity-spending-trends}.} has revealed that the skills gap in information security continues to widen and has doubled in the past five years. In 2014, 23\% of respondents to the survey stated that their organisation had a problematic shortage of information security skills. This had climbed to 51\% at the beginning of this year. Clearly, this is an issue which is being felt across many industries and organisations, and is a concern which extends beyond IT leadership into the boardroom.
\end{quote}
The ESG survey is international, but ESG have confirmed that the UK figures are very similar.
%JHD possibly to write about his recruitment experiences.  ? more IoC input?

In the authors' experience, it is proving very difficult to recruit academic staff with specialisms in Cybersecurity. The demand for Cybersecurity skills in industry makes it difficult for academia to attract academics with knowledge, practical experience, research background and academic aspirations. As universities expand their Cybersecurity provision it is not uncommon to find multiple jobs advertised at the same time.  A recent example had a professor of Cybersecurity, two senior academic positions and 2 junior academic positions in one advert. There are examples in the UK of Cybersecurity lecturing jobs remaining unfilled for longer than a year. There are also examples of  Cybersecurity subject groups moving en masse from one university to another.

%Not sure if we want to raise something about funding of posts ??
%Poaching of staff ??
%Is this too contentious


%{\bf JHD to ? Poll CPHC, and/or survey job adverts.}
\section{Accreditation: RQ3}

For a few years now there has been a recognized need in the UK to build knowledge, skills and capacity in the area of Cybersecurity. This need has led to the establishment of a number of initiatives from a number of national governments for example The UK Cyber Security Strategy. \cite{UKCabinetOffice} or National Initiative for Cybersecurity Education (NICE) in the USA \cite{NICE}. 

The teaching of Cybersecurity in higher education pre-dates these initiatives and there has been recognition of the need for the inclusion of Cybersecurity as part of the Computer Science discipline for a number of years \cite{Hentea2006}. There has  been a debate as to whether Cybersecurity is distinct discipline from Computer Science \cite{McGettreick2013}. The consensus now is that Cybersecurity is both a discipline in its own right and that Cybersecurity should be taught within Computer Science and related degrees. There have been a number of international initiatives international to define curricula to support this for example Computer Science Programmes  \cite[which added ``Information Assurance and Security'' for the first time]{ACM2013a} and specialised Cybersecurity Programmes \cite{ACMIEEEAISSIGSECIFIP}.

In the United Kingdom, Higher Education provision addresses both approaches. A significant number of undergraduate and postgraduate programmes are available in both the areas of Computer Science and Cybersecurity (and closely related fields Computer Security, Digital / Computer Forensics, etc). In the UK, Universities and Colleges Admissions Service (UCAS) lists over 40 Higher Education Institutes (HEI) providing Undergraduate qualifications related to Cybersecurity for entry in September 2019. An even larger number of HEI provide study opportunities related to more general Computer Science. UCAS lists 246 provides for undergraduate programmes related to Computer Science. 

Accreditation has evolved to directly addresses the Cybersecurity challenges in both general Computer Science programmes and specialist Cybersecurity programmes. In the UK, accreditation in the broad computing area is being performed by a few different agencies. These include:

\textbf{1. Not for Profit Organisations}
Tech Partnership Degrees is a Not For Profit Organisation that provides endorsements to Higher Education programmes with specific curricula elements aimed at job market requirements. One of the curricula elements is related to Cybersecurity. Tech Partnership Degrees have a specialist scope, endorsing programmes in the area of IT Management for Business and Software Engineering for Business.  Tech Partnership Degrees currently endorse 14 IT Management for Business Programmes and 5 Software Engineering for Business Programmes. As such Tech Partnership Degrees currently have limited impact upon more general Computer Science education and none upon specialist Cybersecurity education. 

The Institute of Coding is a not for profit organisation that intends to enhance how Digital Skills are developed in Higher Education in the UK. This could potentially include Cybersecurity related skills. Like the Tech Partnership the focus is upon job market requirements.  Given the size of this initiative this potentially has a significant role to play however at time of writing it is not clear what that shall be.

\textbf{2. National Cyber Security Centre (NCSC)}
The NCSC is a UK Government organisation tasked with enhancing the Cybersecurity of the UK. The NCSC publishes and accredits to a number of Cybersecurity standards \cite{NCSC2018a}. These standard are linked to the ACM recommendations for curricula. To date the major focus of NSSC acreditation has been upon Masters degrees specializing in Cybersecurity. More recently the NCSC has also began accrediting integrative masters programmes, undergraduate degrees in Cybersecurity and Computer Science degrees with a significant Cybersecurity focus \cite{NCSC2018b}.

Currently accredited are 15 Cybersecurity MSc programmes with a further 11 provisional accredited, 3 integrated masters Cybersecurity programmes and 1 Cybersecurity Degree Programme with a further 2 provisionally accredited. Hence the extent of accreditation is currently reasonably limited in terms of reach to Computer Science programmes. This appears to be a positive initiative that will hopefully further develop over time. 

\textbf{3. Professional Bodies}
The BCS, The Chartered Institute for IT (BCS) and the Institution of Engineering and Technology (IET) both accredit programmes in the general area of Computer Science and the more specialist area of Cybersecurity discipline areas. The accreditation provided by these institutes are underpinned by international Initiatives such as the Washington Accord\footnote{\url{http://www.ieagreements.org/accords/washington/}.} and Seoul Accord\footnote{\url{https://www.seoulaccord.org/}.}. These memoranda support the internationalising of the curriculum and promote consistency and parity in Computer Science Education globally.   These professional bodies are also registered charities and hence have responsibilities for public good which extends beyond short term job market needs \cite{Stensaker2006}, \cite{Mutereko2017}. Both the IET and The BCS have a long history of expecting coverage environmental factors within the programmes they accredit. The BCS has for a number of years been expecting significant coverage of Legal, Ethical, Social and Professional Issues \cite{Brooke2018}. Clearly Cybersecurity has been and continues to be part of these expectations. 

In recent years the BCS has evolved its Accreditation Practices to promote and mandate the inclusion of Cybersecurity within the programmes the body accredits. The timeline which this process has followed in Table \ref{table:1} .
\begin{center}
  \begin{table}[h!]
  \begin{tabular}{ | p{6cm} |p{1.5cm} |}
    \hline
    IV. ACCREDITATION:RQ3 - A. Developing Expectations &   \\ \hline
    UK Government Cybersecurity Strategy \cite{UKCabinetOffice} & November 2011 \\ \hline
    Three workshops of a consortium of industry, academia and government bodies - led by CPHC and (ISC)2 - leading to the development of Cybersecurity learning guidelines to be embedded into BCS accredited UK Computer Science and IT-related degree \cite{CPHCISC2}  & 2013 to June 2015 \\ \hline
    UK Government report Cybersecurity Skills, Business Perspectives and Government's Next Steps Report Released \cite{UKCabinetOffice2014} & March 2014  \\ \hline
    Council of Professors and Heads of Computing (CPHC) Identifies Cybersecurity as one the top 3 concerns in Computing & April 2014 \\ \hline
    Joint Development of White Paper from CPHC and The International Information Systems Security Certification Consortium(ISC)2 \cite{CPHCISC2014} & April -November 2014 \\ \hline
    Extended Cybersecurity Criteria included in BCS Accreditation Guidelines \cite{BCS2018a}& June 2015 \\
    \hline
    IV. ACCREDITATION:RQ3 - B. What does the BCS tell Universities? & \\ \hline
    Cybersecurity Principles Roadshow & March-April 2016 \\ \hline
    IV. ACCREDITATION:RQ3 - C. Accreditation - what progress has been made? &  \\ \hline
    All visited HEIs expected to be fully compliant & September 2020 (BCS follow a 5-year Accreditation Cycle)  \\ \hline
  \end{tabular}
  \caption{Timeline of the development of Cybersecurity expectations}
  \label{table:1}
  \end{table}
\end{center}
\subsection{Developing Expectations}
Internationally the expectations regarding both the breadth and depth of the expected Cybersecurity coverage has been the subject of much discussion, debate and analysis. Like many Governments, the UK Goverment has been actively been seeking ways to address this \cite{UKCabinetOffice}, \cite{UKCabinetOffice2014}. In parallel to the work complete by the ACM \cite{ACM2013a}, in the UK considerable effort was taken to ensure Industry, Higher Education, Government and the related Professional Bodies collaborated on a set of guidelines which are to the benefit of the various stakeholders and wider society \cite{Irons2016}. In 2013, an initiative was set up by (ISC)2, CPHC (the representative body of Computer Science Departments) and the Cabinet Office to examine embedding Cybersecurity into undergraduate degrees in the UK. Three workshops in 2013 and 2015 attempted to define the principles of Cybersecurity education and proposed a framework for embedding these principles in UK Computing Science curricula. Attendees at the workshops included industry, professional bodies, UK government departments and more than 30 Universities that offer undergraduate Computing Science degrees. This work intially lead to a white paper related to a proposal in the form of a whitepaper \cite{CPHCISC2014}, followed by a set of guidelines \cite{CPHCISC2}. The BCS agreed to adopt the outputs into their accreditation criteria. This was the first time that Cybersecurity has been extensively referenced within accreditation criteria for computing and IT-related degrees. The fact that Cybersecurity is included as a  component of the BCS accreditation criteria, reflects the importance placed on Cybersecurity and the expectation that all computing graduates should have knowledge and skills in Cybersecurity as they move towards Chartered status.

The produced reference guidelines ("Cybersecurity Principles and Learning Outcomes") \cite{CPHCISC2} established a baseline of common knowledge, example learning outcome domains for Cybersecurity within the Computing Science courses and guidance on embedding the concepts. The document provides specific guidance for embedding and enhancing relevant Cybersecurity principles, concepts and learning outcomes within their undergraduate curricula. The document suggested 5 areas of coverage 

\begin{itemize}
    \item Information and risk 
    \item Threats and attacks 
    \item Cybersecurity architecture and operations
    \item Secure systems and products; and 
    \item Cybersecurity management
\end{itemize}

The ambition of this approach is to influence the curricula of all programmes seeking accreditation (regardless of the precise discipline area.)  The approach taken is not intended to be prescriptive or stifle innovation, however it is intended to promote curricula that would benefit the students upon programmes, their future employers and wider society.  In this context this is realized as an expectation Cybersecurity is an inclusion in all degrees accredited by the BCS. e.g. the expectation for coverage is true for Computer Science as well as Cybersecurity programmes. Two criteria are expected to be covered by all programmes seeking accreditation. These are \cite{BCS2018a}:

2.1.6 Recognise the legal, social, ethical and professional issues involved in the exploitation of computer technology and be guided by the adoption of appropriate professional, ethical and legal practices

2.1.9 Knowledge and understanding of information security issues in relation to the design, development and the use of information system

Additionally programmes seeking Chartered Information Technology Professional Accreditation also have to cover:

3.1.2 Knowledge and understanding of methods, techniques and tools for information modelling, management and security

In the context of BCS accreditation, these requirements imply an exit standard that all students on a programme must be able to demonstrate irrespective of the option choices they have made. This means a HEI applying for accreditation is expected to provide evidence that the criteria are taught and assessed in a non-trivial manner, to and by all students upon the programme seeking accreditation . A HEI is expected to provide evidence in the form of programme and module specification documentation and example assessment specifications (coursework and examinations). These criteria and the expectation that they are taught and assessed has been present for a number of years.

The BCS accreditation (2.1.6, 2.1.9 and 3.1.2) is not prescriptive, but encourages HEIs to embed Cybersecurity teaching across a range of subject areas in the computer science curriculum such as programming, software design, databases, networking, architecture. In addition there an expectation that there is significant coverage of Cybersecurity principles and fundamentals - either as a stand alone module or as a significant component(s) of other modules. This approach differs from the ACM approach where the expectations are more explicit and the curriculum expectations are specified at a more granular level.


\subsection{What does BCS tell Universities}
The agreed Cybersecurity Principles and Learning Outcome \cite{CPHCISC2} were discussed with the wider education community by a road-show led by CPHC. A series of workshops took place in 2015 which presented the rationale for embedding Cybersecurity in the curriculum of Computing Science degrees. The workshops included case studies from universities who had embedded Cybersecurity Computer Science curriculum illustrating different approaches to implementation. The workshops had 102 attendees from the academic Computing Science community representing 60 UK Universities. 

The BCS Guidelines on Course Accreditation are published upon the BCS website \cite{BCS2018a}. The BCS also publishes the changes that have been made \cite{BCS2018b}. When changes are made, the BCS communicates the changes by email and in writing to all the BCS Educational Affiliates, that is all the HEIs that seek accreditation from the BCS. The expectations for Cybersecurity were extended in the June 2015 version of the guidelines for consideration at Accreditation Visits that took place from September 2015 or later.

This change to the accreditation guidelines is now in an implementation period. The accreditation process adopted by the BCS is cyclic in nature. Formally, the cycle is 5 years in duration. The new expectations have been implemented as follows. To ensure continuous accreditation, accreditation visits are normally scheduled every 5 years. At the time of the next visit in this accreditation cycle, accreditation is conditional upon a HEI having considered the guidelines and either adjusted the curriculum to meet the new expectations or have a formal plan in place for when and how adjustments will be made.  It is anticipated that from 2020 the expectation will be all accredited programmes have the new expectations fully embedded.

In the year prior to an Accreditation Visit, HEIs are invited to an Accreditation Briefing from the BCS. The intention of the briefing is to help ensure Accreditation Visits go smoothly from the perspective of both the BCS and visited HEI. The briefings take place virtually. The briefing includes a summary of the process, discussion of recent changes, guidance regarding the application and a summary of common issues that are being seen in other HEIs. Significant opportunity for seeking clarification is provided. One of the issues that is highlighted is not all institutions have yet evolved their programmes to fully address the increased expectation for Cybersecurity. This is resulting in accreditation being contingent upon a HEI taking action to address this short fall or in some cases the withdrawal of accreditation. A number of HEIs are in the process of adjusting their curricula to meet the new expectations. In this case, the BCS notes the changes to programme design, the outputs from which will be scrutinized at the next accreditation visit.

\subsection{Accreditation - what progress has been made}

This initiative is a collective attempt to formally include Cybersecurity in all BCS Accredited programmes. Some of these programmes will be specialist Cybersecurity programmes, however the majority will take a different emphasis. This is a work in progress. A full cycle of accreditation visits has not yet taken place following the adjustment to the BCS Guidelines. What is being observed is the majority of visited HEIs have now either adjusted their curricula to extend the coverage of Cybersecurity or have a plan in place to do so. However, a minority are requiring encouragement to do so.

From the start of the Autumn 2015 term, up to and including the Autumn 2018 term, the BCS have carried out 70 accreditation visits (including 4 international visits). The BCS identified action was required to address concerns related to Cybersecurity at 16 HEIs. So 54 HEIs were already delivering Cybersecurity in keeping with the BCS expectations.

Long term actions were expected from 12 HEIs (6 in 2015/16, 3 in 2016/17 and 3 in the Autumn of 2018) who were awarded 'At Threshold' judgments. 10 of these judgments were across all programmes. 1 was specifically against a generalist masters programme only. This indicates that the BCS will expect adjustments to have taken place before the next accreditation visit. As indicated earlier, this was commonly the case that adjustments had been made to approved programmes of study, however the adjusted modules had not yet been delivered so the evidence base was incomplete in terms of how Cybersecurity was assessed.
 
Short terms 90 Day Responses where required from 4 HEIs. The outcomes of these actions were as follows:
                                                                        
A)	Of the 11 UG programmes involved all were approved 'At Threshold'

B)	Of the 9 UG programmes involved, 8 were approved and 1 refused

C)	Of the 5 UG programmes involved, all approved 'At Threshold'

D)	Of the 3 UG programmes involved, all 3 were refused.
 
Good practice was identified at one university by the commendation:
 \begin{quote}
``The second year project provides an opportunity for exploring security aspects in depth with an industrial use case''.
\end{quote}
In sum, this shows that many UK HEIs have now embedded Cybersecurity in their provision, a number are in the process of doing so and a minority have chosen not to. Clearly not all HEIs in the UK necessarily have to apply for accreditation, or apply for accreditation for all their programmes, but even so this is significant evidence of inclusion of Cybersecurity to an agreed standard.

\section{Conclusions}
As regards our research questions, we can make the following comments.
\begin{description}
\item[RQ1]The guidelines from both ACM and BCS are good for general education. However, the most important item would seem to be an attitude of caution with respect to both offline (\S\ref{sec:SQL}) and online (\S\ref{sec:Java}--) resources. 
\item[RQ2]The recommendation in \cite[p. 98]{ACM2013a} that Cybersecurity be taught largely through other Knowledge Areas is, in abstract, a good idea. Indeed, a recnt discussion on Cybersecurity had the aerospace engineers amazed that security was a separate topic: ``for us, safety is present everywhere''.  However, in the current state of education resources (sections \S\ref{sec:SQL}--\ref{sec:Java}) this may be a counsel of perfection.  It is more important that issues like SQL injection \cite{Drop2019} or correct use of SSL/TLS \cite{Chenetal2019a} be taught somewhere than tthat they not be taught at all.
\par
We commend the BCS-identified good practice of exploring Cybersecurity via a project, possibly in PCI DSS (\S\ref{sec:PCIDSS}).
\item[RQ3]Accreditation (as practised by BCS) is a valuable tool in improving the standard of Cybersecurity teaching, and spreading good practice, and should continue this.
\end{description}

We have the following specific recommendations.
\begin{enumerate}
\item Database courses should look carefully at the security aspects of the texts they use, and the examples they quote, on the lines of \cite{Drop2019}.
\item All computer science courses should emphasise that informal resources should come with a ``security health warning'': see sections \ref{sec:informal} and \ref{sec:Java}. One should probably use the data from \cite{Chenetal2019a}: ``If you pick up a SSL/TLS answer from Stack Overflow, there's a 70\% chance it's insecure''.
\end{enumerate}

\section*{Acknowledgements}
Thanks to Sally Pearce, Academic Accreditation Manager, BCS--The
Chartered Institute for IT for supplying the summary information
related to Accreditation.

\bibliographystyle{IEEEtran}
\bibliography{FIE2019} 

\end{document}
