%%%% Proceedings format for most of ACM conferences (with the exceptions listed below) and all ICPS volumes.
\documentclass[sigconf]{acmart}
\usepackage{paralist}
\usepackage{url}
\usepackage[hyphenbreaks]{breakurl}

\def\UrlBreaks{\do\/\do-}

%%%% As of March 2017, [siggraph] is no longer used. Please use sigconf (above) for SIGGRAPH conferences.

%%%% Proceedings format for SIGPLAN conferences 
% \documentclass[sigplan, anonymous, review]{acmart}

%%%% Proceedings format for SIGCHI conferences
% \documentclass[sigchi, review]{acmart}

\usepackage{booktabs} % For formal tables

% Copyright
%\setcopyright{none}
%\setcopyright{acmcopyright}
\setcopyright{acmlicensed}
%\setcopyright{rightsretained}
%\setcopyright{usgov}
%\setcopyright{usgovmixed}
%\setcopyright{cagov}
%\setcopyright{cagovmixed}

\copyrightyear{2019}
\acmYear{2019}
\setcopyright{acmlicensed}
\acmConference[ITiCSE '19]{The 24th Annual Conference on Innovation
  and Technology in Computer Science Education}{Jul. 15-17,
  2019}{Aberdeen, UK}
%\acmBooktitle{}
%\acmPrice{15.00}
%\acmDOI{10.1145/3159450.3159547}
%\acmISBN{978-1-4503-5103-4/18/02}
% This slight change to the code may also save 1 or 2 lines of space.

% removes the headers from each page per the preparation instructions, as these are not needed and will be updated with the chairs' actual session names during the pagination/indexing process:
\fancyhead{}

\begin{document}
\title{How Well do we Teach Cybersecurity?}
%\title{What Do We Mean By Cybersecurity?}  Tom's original was this
%\titlenote{}
%\subtitle{Extended Abstract}
%\subtitlenote{}
\author{James H. Davenport}
\orcid{0000-0002-3982-7545}
\affiliation{%
  \institution{University of Bath}
  \streetaddress{}
  \city{Bath} 
  \country{United Kingdom}
}
\email{j.h.davenport@bath.ac.uk}

\author{Tom Crick}
\orcid{0000-0001-5196-9389}
\affiliation{%
  \institution{Swansea University}
  \streetaddress{}
  \city{Swansea} 
  \country{United Kingdom}
}
\email{thomas.crick@swansea.ac.uk}

 
% The default list of authors is too long for headers}
%\renewcommand{\shortauthors}{Moller and Crick}


\begin{abstract}
Abstract here...
\end{abstract}


\keywords{TBC}

\maketitle


\section{Introduction}

Cybersecurity has been in the news for several years, generally prompted by spectacular breaches of one kind or another, such as \cite{BritishAirways2018a}.

There are calls for education to respond to this situation, which it does both at the individual level and via recommended curricula \cite{ACM2013a} and professional accreditation requirements \cite{BCS2018a}.

Nevertheless, it is one thing to write curricula and requirements, and
another thing to deliver appropriate education, and one could
reasonably ask how well this is done in practice.

\section{Policy Context}

Substantial CS/digital curriculum reform across the
UK~\cite{crick+sentance:2011,brown-et-al:sigcse2013,wgictreview:2013,brown-et-al:toce2014,moller+crick:jce2018}
-- but what about focus on cybersecurity?

UK national economic skills priority
e.g. \url{https://www.gov.uk/government/publications/national-cyber-security-strategy-2016-to-2021}
(esp. \url{https://publications.parliament.uk/pa/jt201719/jtselect/jtnatsec/706/70605.htm}
and \url{https://www.ncsc.gov.uk/blog/skills-and-training} and
\url{https://www.cybersecuritychallenge.org.uk/}). 

There is large-scale media
attention on the ``cybersecurity skills gap''
e.g. \url{https://www.contracts.mod.uk/do-features-and-articles/digital-skills-shortage-threatens-uk-cyber-security/}
and
\url{https://www.itpro.co.uk/cyber-security/31554/uk-government-lacks-urgency-in-tackling-cyber-security-skills-gap}
etc -- link to UK Digital Strategy and UK Industrial Strategy.

\section{Challenges}

% should we frame this as one or more case studies?
% do we have any national survey data?

\subsection{PCI DSS}
\cite{PCI2018b}.

\subsection{Educational Resources}
JHD to write about Database books.
\cite{Guimaraesetal2004}

\subsection{Staff}
JHD possibly to write about his recruitment experiences.  ? more IoC input?

\subsection{Accreditation}
?AI/TP to write something.

%\section*{Acknowledgements}


\bibliographystyle{ACM-Reference-Format}
\bibliography{ITiCSE2019} 

\end{document}
