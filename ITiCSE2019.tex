%%%% Proceedings format for most of ACM conferences (with the exceptions listed below) and all ICPS volumes.
\documentclass[sigconf]{acmart}
\usepackage{paralist}
\usepackage{url}
\usepackage[hyphenbreaks]{breakurl}

\def\UrlBreaks{\do\/\do-}

%%%% As of March 2017, [siggraph] is no longer used. Please use sigconf (above) for SIGGRAPH conferences.

%%%% Proceedings format for SIGPLAN conferences 
% \documentclass[sigplan, anonymous, review]{acmart}

%%%% Proceedings format for SIGCHI conferences
% \documentclass[sigchi, review]{acmart}

\usepackage{booktabs} % For formal tables

% Copyright
%\setcopyright{none}
%\setcopyright{acmcopyright}
\setcopyright{acmlicensed}
%\setcopyright{rightsretained}
%\setcopyright{usgov}
%\setcopyright{usgovmixed}
%\setcopyright{cagov}
%\setcopyright{cagovmixed}

\copyrightyear{2019}
\acmYear{2019}
\setcopyright{acmlicensed}
\acmConference[ITiCSE '19]{The 24th Annual Conference on Innovation
  and Technology in Computer Science Education}{Jul. 15-17,
  2019}{Aberdeen, UK}
%\acmBooktitle{}
%\acmPrice{15.00}
%\acmDOI{10.1145/3159450.3159547}
%\acmISBN{978-1-4503-5103-4/18/02}
% This slight change to the code may also save 1 or 2 lines of space.

% removes the headers from each page per the preparation instructions, as these are not needed and will be updated with the chairs' actual session names during the pagination/indexing process:
\fancyhead{}

\begin{document}
\title{Are We Teaching Cybersecurity (and how well)?\\(Abstract 20 January, Paper 27th)}
%\title{What Do We Mean By Cybersecurity?}  Tom's original was this
%\titlenote{}
%\subtitle{Extended Abstract}
%\subtitlenote{}
\author{James H. Davenport}
\orcid{0000-0002-3982-7545}
\affiliation{%
  \institution{University of Bath}
  \streetaddress{}
  \city{Bath} 
  \country{United Kingdom}
}
\email{j.h.davenport@bath.ac.uk}

\author{Tom Crick}
\orcid{0000-0001-5196-9389}
\affiliation{%
  \institution{Swansea University}
  \streetaddress{}
  \city{Swansea} 
  \country{United Kingdom}
}
\email{thomas.crick@swansea.ac.uk}

\author{Alastair Irons}
\orcid{}
\affiliation{%
  \institution{Sunderland University}
  \streetaddress{}
  \city{Sunderland} 
  \country{United Kingdom}
}
\email{alastair.irons@sunderland.ac.uk}

\author{Tom Prickett}
\orcid{}
\affiliation{%
  \institution{Northumbria University}
  \streetaddress{}
  \city{Newcastle upon Tyne} 
  \country{United Kingdom}
}
\email{tom.prickett@northumbria.ac.uk}

 
% The default list of authors is too long for headers}
%\renewcommand{\shortauthors}{Moller and Crick}


\def\IntroYear{2014}  % when BCS started asking for CyberSecurity
\begin{abstract}
There are many facets to CyberSecurity education: we address the questions ``what should the generalist computer scientist know'' and ``how well is that being taught''. We consider particularly the UK perspective, where the British Computer Society accredits computing degrees, and (since \IntroYear) insists on a certain CyberSecurity element in all such.
\end{abstract}


\keywords{TBC}

\maketitle

\section{Introduction}

Cybersecurity has been in the news for several years, generally prompted by spectacular breaches of one kind or another, such as \cite{BritishAirways2018a}.
It needs attention across the spectrum:
\begin{quote}
change the culture in your organisation around cyber security; to t ry to do for cyber what has been done so successfully for health and safety, for example, over the last ten years - to get everybody to take it seriously; to take the risk management process seriously and drive that down through the organisation. \cite{Hannigan2019a}
\end{quote}

There are calls for education to respond to this situation, which it does both at the individual level and via recommended curricula \cite{ACM2013a} and professional accreditation requirements \cite{BCS2018a}. There has been a recent international working group \cite{Parrishetal2018a}, but this has yet to report.

Nevertheless, it is one thing to write curricula and requirements, and
another thing to deliver appropriate education, and one could
reasonably ask how well this is done in practice.
\subsection{CyberSecurity: for all or for specialists?}
In one sense, this title is a false dichotomy: there is a serious need for cybersecurity specialists (estimates vary, but are always large: \cite{JCNSS2018a} has only anecdotal evidence), but also all in IT need to know \emph{some} CyberSecurity, as the recent fashion for talking about DevSecOps rather than just DevOps exemplifies.

This is not a brand-new concern: see \cite{Parr2014a}

AI: GDPR essentially expects CyberSecurity to be in all coding. 
JHD: PCI DSS insists on a Web Application Firewall, which is only needed if you don't trust your applications.


Specialist curricula abound: a recent one is \cite{ACMIEEEAISSIGSECIFIP}. However, given the shortage of specialists, in practice a good generalist with some CyberSecurity expertise can often get a ``specialist'' job.
\subsection{Research Questions}
\begin{enumerate}
\item The person in the street --- probably not. ({\emph{interesting point from a public engagement/science communication perspective -- aren't ``we'' trying to create a digitally competent and capable citizenry? Minor point but maybe worth mentioning e.g. being safe online, accessing public services, etc}})
\item[JHD]This is important, agreed, but I think this is being a user: think car driving rather than car design. BUT the point I heard from the WC Engineers ``how can security be an option --- safety certainly isn't'' is also relevant.
\item The general CS graduate.
\item The general CS masters graduate.
\item The specialist CS graduate.
\item The specialist CS masters graduate.
\end{enumerate}
\section{Policy Context}

Substantial CS/digital curriculum reform across the
UK~\cite{crick+sentance:2011,brown-et-al:sigcse2013,wgictreview:2013,brown-et-al:toce2014,moller+crick:jce2018}
-- but what about focus on cybersecurity?

UK national economic skills priority
e.g. \url{https://www.gov.uk/government/publications/national-cyber-security-strategy-2016-to-2021}
%(esp. \url{https://publications.parliament.uk/pa/jt201719/jtselect/jtnatsec/706/70605.htm}
(especially \cite{JCNSS2018a}
and \url{https://www.ncsc.gov.uk/blog/skills-and-training} and
\url{https://www.cybersecuritychallenge.org.uk/}). 

There is large-scale media
attention on the ``cybersecurity skills gap''
e.g. \url{https://www.contracts.mod.uk/do-features-and-articles/digital-skills-shortage-threatens-uk-cyber-security/}
and
\url{https://www.itpro.co.uk/cyber-security/31554/uk-government-lacks-urgency-in-tackling-cyber-security-skills-gap}
etc -- link to UK Digital Strategy and UK Industrial Strategy.

\section{Challenges}

What should be in the general curriculum.  Pervasive (as in Maths or LSEPI) rather than specialist option.  
{\bf JHD to research ACM general curriculum; also what is status of CYBOOK etc.}

Is teaching CyberSecurity different? Lecturing is probably not the best way. Should we use real-life case studies (e.g. British Airways)

AI noted the workshops with ISC$^2$, with the ``this is a fad'' concept.
{\bf AI to write up}.
  Should we be teaching it because it's underpinning? because there's a skill shortage?  JHD would argue both.

% should we frame this as one or more case studies?
% do we have any national survey data?

\subsection{PCI DSS}
\cite{PCI2018b}.

Alastair to write about their Credit Fraud experience (MSc CyberSecurity).
HE Academy grant (to be referenced).


\subsection{Educational Resources: SQL Injection}
It is 15 years since \cite{Guimaraesetal2004} 
wrote ``All the topics listed above should be presented in the first
Database Course'', and the first such topic was SQL injection \cite{SPIDynamics2002,Anonymous2018b}. SQL injection as an attack has been around for twenty years \cite{HornerHyslip2017a}, has its own cartoon (\url{https://xkcd.com/327/}, dating back to 2007 according to the Internet Archive) and website (\url{http://bobby-tables.com/}). Nevertheless SQL injection is still a major weakness: number one in the OWASP Top 10 \cite{OWASP2017a}, and has been in the Top 10 since at least 2003. 

Clearly such a major weakness should be well-taught. One would like to think it is, but the first author has yet to meet a UK undergraduate who recalls being taught it \emph{outside} a specialist cyber-security course.  In general it's hard to determine what is taught, but a reasonable proxy for this is the content of recomended textbooks. 

Hence \cite{Drop2019} analyzed the database textbooks used by  44 of the top 50 Computer Science
departments in the United States (using \cite{StangerMartin2015a}'s list, the other six didn't have a book listed). There were seven such boks, but three accounted for the 36 of the 44. Five of the seven (30 of the 44) had no mention of SQL injection. On the other two, the more popular one has a seriously flawed discussion\footnote{``However, they imply that using parameters is equivalent to using a function to add escape characters
around user input. This is incorrect, as using parameters allows
SQL statements to be pre-compiled, and prevents any user input
from being interpreted as code, while escaping user input is not
recommended as a sole defense since imperfect escape functions
can easily be subverted.'' \cite{Drop2019}}, and the other, while generally excellent, had a presentational problem\footnote{``However, the fact that the first
example should not be used is not discussed until two pages after
the example in the text, and is not mentioned at all in the caption or
on the page where the figure appears. This means a student who is
skimming the text looking for an example to modify for their own
code could simply copy the code that first appears in the example,
without being aware that this is in fact an example of what they
shouldn't do.'' \cite{Drop2019}}.

\subsection{Informal Resources}
The web abounds with informal resources: tutorials and code snippets. How good are these, and how good are people at using these? This has been looked at by \cite{Unruhetal2017a}, who took the top five results from Google for six queries. Of these 30 tutorials, 6 had SQL injection weaknesses, and 3 had Cross-Site Scripting\footnote{Number 7 in WASP's Top Ten \cite{OWASP2017a}.} weaknesses. Searching for these fragments in PHP projects GitHub found 82 instances of these fragments, of which 117 were verified by manual to be vulnerable --- 80\% of which were vulnerable to SQL injection.

\cite{Mengetal2018a,Chenetal2019a,Zorz2019a}

\subsection{Staff}
It is well known that CyberSecurity skills are in short supply, for example \cite{Page2018a}.
\begin{quote}
% JHD: I am in touch with ESG to try to find more, and someUK-specific numbers.
Research into the state of IT conducted annually by ESG\footnote{Apparently \url{https://www.esg-global.com/research/esg-brief-2018-cybersecurity-spending-trends}.} has revealed that the skills gap in information security continues to widen and has doubled in the past five years. In 2014, 23\% of respondents to the survey stated that their organisation had a problematic shortage of information security skills. This had climbed to 51\% at the beginning of this year. Clearly, this is an issue which is being felt across many industries and organisations, and is a concern which extends beyond IT leadership into the boardroom.
\end{quote}
\begin{quote}

\end{quote}
JHD possibly to write about his recruitment experiences.  ? more IoC input?

{\bf Alastair's experiences.}

{\bf JHD to ? Poll CPHC, and/or survey job adverts.}
\subsection{Accreditation}
{\bf{Alastair could you check this and adjust if appropriate?}}

For a few years now there has been a recognized need to build knowledge, skills and capacity in the area of Cybersecurity. This need has led to the establishment of a number of initiatives from a number of national governments for example The UK Cyber Security Strategy. \cite{UKCabinetOffice} or National Initiative for Cybersecurity Education (NICE) in the USA \cite{NICE}. 

The teaching of Cybersecurity in higher education pre-dates these initiatives and there has been recognition of the need for the inclusion of Cybersecurity as part of the Computer Science discipline for a number of years \cite{Hentea2006}. There has  been a debate as to whether Cybersecurity is distinct discipline from Computer Science \cite{McGettreick2013}. The consensus now is that Cybersecurity is both a discipline in its own right and that Cybersecurity should be taught within Computer Science and related degrees. There have been a number of international initiatives international to define curricula to support this for example Computer Science Programmes  \cite{ACM2013a} and specialised Cybersecurity Programmes \cite{ACMIEEEAISSIGSECIFIP}.

In the United Kingdom, Higher Education provision addresses both approaches. A significant number of undergraduate and postgraduate programmes are available in both the areas of Computer Science and Cybersecurity ( and closely related fields Computer Security, Digital / Computer Forensics, etc). In the UK, Universities and Colleges Admissions Service (UCAS) lists over 40 Higher Education Institutes (HEI) providing Undergraduate qualifications related to Cybersecurity for entry in September 2019. An even larger number of HEI provide study opportunities related to more general Computer Science. UCAS lists 246 provides for undergraduate programmes related to Computer Science. 

Accreditation has evolved to directly addresses the Cybersecurity challenges in both general Computer Science programmes and specialist Cybersecurity programmes. In the UK, accreditation in the broad computing area is being performed by a few different agencies. These include:

\textbf{1. Not for Profit Organisations}
Tech Partnership Degrees is a Not For Profit Organisation that provides endorsements to Higher Education programmes with specific curricula elements aimed at job market requirements. One of the curricula elements is related to Cybersecurity. Tech Partnership Degrees have a specialist scope, endorsing programmes in the area of IT Management for Business and Software Engineering for Business.  Tech Partnership Degrees currently endorse 14 IT Management for Business Programmes and 5 Software Engineering for Business Programmes. As such Tech Partnership Degrees currently have limited impact upon more general Computer Science education and none upon specialist Cybersecurity education. 

The Institute of Coding is a not for profit organisation that intends to enhance how Digital Skills are developed in Higher Education in the UK. This could potentially include Cybersecurity related skills. Like the Tech Partnership the focus is upon job market requirements.  Given the size of this initiative this potentially has a significant role to play however at time of writing it is not clear what that shall be.

\textbf{2. National Cyber Security Centre (NCSC)}
The NCSC is a UK Government organisation tasked with enhancing the Cybersecurity of the UK. The NCSC publishes and accredits to a number of Cybersecurity standards \cite{NCSC2018a}. These standard are linked to the ACM recommendations for curricula. To date the major focus of NSSC acreditation has been upon Masters degrees specializing in Cybersecurity. More recently the NCSC has also began accrediting integrative masters programmes, undergraduate degrees in Cybersecurity and Computer Science degrees with a significant Cybersecurity focus \cite{NCSC2018b}.

Currently accredited are 15 Cybersecurity MSc programmes with a further 11 provisional accredited, 3 integrated masters Cybersecurity programmes and 1 Cybersecurity Degree Programme with a further 2 provisionally accredited. Hence the extent of accreditation is currently reasonably limited in terms of reach to Computer Science programmes. This appears to be a positive initiative that will hopefully further develop over time. 

\textbf{3. Professional Bodies}
The BCS, The Chartered Institute for IT (BCS) and the Institution of Engineering and Technology (IET) both accredit programmes in the general area of Computer Science and the more specialist area of Cybersecurity discipline areas. The accreditation provided by these institutes are underpinned by international Initiatives such as the Washington Accord and Seoul Accord (do we need to reference these). These memorandums support the internationalising of the curriculum and promote consistency and parity in Computer Science Education globally.   These professional bodies are also registered charities and hence have responsibilities for public good which extends beyond short term job market needs \cite{Stensaker2006}, \cite{Mutereko2017}. Both the IET and The BCS have a long history of expecting coverage environmental factors within the programmes they accredit. The BCS has for a number of years been expecting significant coverage of Legal, Ethical, Social and Professional Issues \cite{Brooke2018}. Clearly Cybersecurity has been and continues to be part of these expectations. 

This ambition of this approach is to influence the curricula of all programmes seeking accreditation (regardless of the precise discipline area.)  The approach taken is not intended to be prescriptive or stifle innovation, however it is intended to promote curricula that would benefit the students upon programmes, their future employers and wider society.  In this context this is realized as an expectation Cybersecurity is an inclusion in all degrees accredited by the BCS. e.g. the expectation for coverage is true for Computer Science as well as Cybersecurity programmes. Two criteria are expected to be covered by all programmes seeking accreditation. These are \cite{BCS2018a}:

2.1.6 Recognise the legal, social, ethical and professional issues involved in the exploitation of computer technology and be guided by the adoption of appropriate professional, ethical and legal practices

2.1.9 Knowledge and understanding of information security issues in relation to the design, development and the use of information system

Additionally programmes seeking Chartered Information Technology Professional Accreditation also have to cover:

3.1.2 Knowledge and understanding of methods, techniques and tools for information modelling, management and security

In the context of BCS accreditation, these requirements imply an exit standard that all students on a programme must be able to demonstrate irrespective of the option choices they have made. This means a HEI applying for accreditation is expected to provide evidence that the criteria are taught and assessed in a non-trivial manner, to and by all students upon the programme seeking accreditation . A HEI is expected to provide evidence in the form of programme and module specification documentation and example assessment specifications (coursework and examinations). These criteria and the expectation that they are taught and assessed has been present for a number of years.

Internationally the expectations regarding both the breadth and depth of the expected Cybersecurity coverage has been the subject of much discussion, debate and analysis. Building on the work complete by the ACM \cite{ACM2013a}, in the UK considerable effort was taken to ensure Industry, Higher Education, Government and the related Professional Bodies collaborated on a set of guidelines which are to the benefit of the various stakeholders and wider society \cite{Irons2016}. This development was a partnership between the BCS,  The Council of Heads and Professors of Computing, (ISC)2 (the largest not-for-profit membership body of certified information and software security professionals worldwide, with over 130,000 members) and the Office for Cyber Security and Information Assurance in the Cabinet Office. This project lead to the agreement of a set of Cybersecurity Principles and Learning Outcomes for Computer Science and related IT Degrees in the UK \cite{CPHCISC2}. These guidelines are closely tied to  the ACM recommendations for curricula \cite{ACM2013a}.

The agreed Cybersecurity Principles and Learning Outcomes were included in the BCS Guidelines for accreditation in 2015. This resulted in an extended expectation of the coverage of Cybersecurity in programmes accredited by the Institute. The expectation is now that all students upon all accredited programmes will graduate with skills, knowledge and experience related to:
\begin{enumerate}
\item Information and risk
\item Threats and attacks
\item Cybersecurity architecture and operations
\item Secure systems and products
\item Cybersecurity management
\end{enumerate}



\subsection{What does BCS tell Universities}
The agreed Cybersecurity Principles and Learning Outcome were discussed with the wider education community by a road-show lead by CPHC. {\bf{(Alastair could you add something here?)}}

The BCS Guidelines on Course Accreditation are published upon the BCS website \cite{BCS2018a}. The BCS also publishes the changes that have been made \cite{BCS2018b}. When changes are made, the BCS communicates the changes by email and in writing to all the BCS Educational Affiliates, that is all the HEIs that seek accreditation from the BCS. The expectations for Cybersecurity were extended in the June 2015 version of the guidelines for consideration at Accreditation Visits that took place from September 2015 or later.

This change to the accreditation guidelines is now in an implementation period. The accreditation process adopted by the BCS is cyclic in nature. Formally, the cycle is 5 years in duration. The new expectations have been implemented as follows. To ensure continuous accreditation, accreditation visits are normally scheduled every 5 years. At the time of the next visit in this accreditation cycle, accreditation is conditional upon a HEI having considered the guidelines and either adjusted the curriculum to meet the new expectations or have a formal plan in place for when and how adjustments will be made.  It is anticipated that from 2020(?) the expectation will be all accredited programmes have the new expectations fully embedded.

In the year prior to an Accreditation Visit, HEIs are invited to an Accreditation Briefing from the BCS. The intention of the briefing is to help ensure Accreditation Visits go smoothly from the perspective of both the BCS and visited HEI. The briefings take place virtually. The briefing includes a summary of the process, discussion of recent changes, guidance regarding the application and a summary of common issues that are being seen in other HEIs. Significant opportunity for seeking clarification is provided. One of the issues that is highlighted is not all institutions have yet evolved their programmes to fully address the increased expectation for Cybersecurity. This is resulting in accreditation being contingent upon a HEI taking action to address this short fall or in some cases the withdrawal of accreditation. A number of HEIs are in the process of adjusting their curricula to meet the new expectations. In this case, the BCS notes the changes to programme design, the outputs from which will be scrutinized at the next accreditation visit.

\subsection{Accreditation - what progress has been made}

This initiative is a collective attempt to formally include Cybersecurity in all BCS Accredited programmes. Some of these programmes will be specialist Cybersecurity programmes, however the majority will take a different emphasis. This is a work in progress. A full cycle of accreditation visits has not yet taken place following the adjustment to the BCS Guidelines. What is being observed is the majority of visited HEIs have now either adjusted their curricula to extend the coverage of Cybersecurity or have a plan in place to do so. However, a minority are requiring encouragement to do so.

From the start of the Autumn 2015 term, up to and including the Autumn 2018 term, the BCS have carried out 70 accreditation visits (including 4 international visits). The BCS identified action was required to address concerns related to Cybersecurity at 16 HEIs. So 54 HEIs were already delivering Cybersecurity in keeping with the BCS expectations.

Long term actions were expected from 12 HEIs (6 in 2015/16, 3 in 2016/17 and 3 in the Autumn of 2018) who were awarded 'At Threshold' judgments. 10 of these judgments were across all programmes. 1 was specifically against a generalist masters programme only. This indicates that the BCS will expect adjustments to have taken place before the next accreditation visit. As indicated earlier, this was commonly the case that adjustments had been made to approved programmes of study, however the adjusted modules had not yet been delivered so the evidence base was incomplete in terms of how Cybersecurity was assessed.
 
Short terms 90 Day Responses where required from 4 HEIs. The outcomes of these actions were as follows:
                                                                        
1)	Of the 11 UG programmes involved all were approved 'At Threshold'

2)	Of the 9 UG programmes involved, 8 were approved and 1 refused

3)	Of the 5 UG programmes involved, all approved 'At Threshold'

4)	Of the 3 UG programmes involved, all 3 were refused.
 
Good practice was identified at one university by the commendation:
 
"The second year project provides an opportunity for exploring security aspects in depth with an industrial use case"

In summation, this shows that many UK HEIs have now embedded Cybersecurity in their provision, a number are in the process of doing so and a minority have chosen not to. Clearly not all HEIs in the UK necessarily have to apply for accreditation, or apply for accreditation for all their programmes, but even so this is significant evidence of inclusion of Cybersecurity to an agreed standard.


\section*{Acknowledgements}
Thanks to Sally Pearce, Academic Accreditation Manager, BCS, The Chartered Institute for IT for supplying the summary information related to Accreditation.

\bibliographystyle{ACM-Reference-Format}
\bibliography{ITiCSE2019} 

\end{document}
