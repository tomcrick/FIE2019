\font\manual=manfnt
\def\dbend{{\manual\char127}}
\def\eqq{{\buildrel?\over=}}
\def\C{\mathbf{ C}}
\def\N{\mathbf{ N}}
\def\Q{\mathbf{ Q}}
\def\R{\mathbf{ R}}
\def\Z{\mathbf{Z}}
\def\F{\mathbf{F}}
\def\softO{\tilde{\cal O}}
\def\O{{\cal O}}
%\def\N{{\bf N}}
%\def\Q{{\bf Q}}
%\def\R{{\bf R}}
%\def\Z{{\bf Z}}
\def\action#1{\hfil\rlap{\bf#1}\break}
\def\pcite#1{[#1]}
\def\Res{\mathop{\rm Res}\nolimits}
\def\r#1#2{``#1''$\rightarrow$``#2''}
\documentclass[a4paper,11pt]{article}
\usepackage[top=1.8cm,bottom=1.8cm,left=1.8cm,right=1.8cm,asymmetric]{geometry}
\bibliographystyle{alphaurl}
\usepackage[hyphens]{url}
%\usepackage{enumitem}
\usepackage{pdfpages}
\usepackage{verbatim}
\usepackage{hyperref}
\usepackage[show]{ed}
\usepackage{graphicx}
\newtheorem{proposition}{Proposition}
\newtheorem{theorem}{Theorem}
\newtheorem{lemma}{Lemma}
\newtheorem{corollary}{Corollary}
\newtheorem{definition}{Definition}
\newtheorem{notation}{Notation}
\newtheorem{example}{Example}
\newtheorem{problem}{Problem}
\def\decision#1{\par{\bf #1}}
\pagestyle{empty}
\begin{document}
\author{Crick/Davenport/Irons/Prickett}
\title{Response to Referees for FIE 2019 Paper \#619 (1570524395):\\``A
  UK Case Study on Cybersecurity Education and Accreditation''}
\date{22 June 2019}
\maketitle

\section*{Summary}

In general, alongside addressing the main points from the referees,
the paper has been substantially restructured and refined for clarity
of its main message, as well as overall presentation and
readability. This has included a new title, refreshed abstract,
restructured introductory section, restructured conclusions, as well
as updating/tidying of the references.

\section*{Referee 1}

All addressed/no comments that require addressing in detail.

\section*{Referee 2}

All addressed/no comments that require addressing in detail.

\section*{Referee 3}

\begin{enumerate}
\item ``{\emph{The research appears to be poorly structured and the analysis/argument is hard to interpret}}''.
\item[A]The paper has been re-written to make the structure
clearer. In particular, the abstract has been updated to more
succinctly provide the essence of the work; the signposting has been
enhanced to better guide the reader; a research approach section to
more clearly explain the approach followed has been added; the
research questions have been made more precise to communicate the
intentions more clearly; and the sections have been made more
self-contained and logically structured to better support the reading
of the paper in a non-linear manner.

\item ``{\emph{The abstract is too long and could be made concise. For example, the 2nd paragraph can be avoided as the main idea has been conveyed in the 1st paragraph as well.}}''
\item[A] The abstract is now two paragraphs long and better represents
  the main thrust of the work.

\item {\emph{What do the authors mean by ``There has been a recent international working group but this has yet to report.''}} 
\item[A] In parallel to the research explored in the paper, there is a
  working group as part of the ACM conference series on Innovation and Technology in Computer Science
  Education (ITiCSE) intending to develop a taxonomy of cybersecurity
  education. This has been clarified in Section 1 of the paper.

\item {\emph{It would add more clarity to introduce what these terms DevOps vs DevSecOps represent in the paper. Also, please use the acronym GDPR after introducing it during the first usage such as General Data Protection Regulation (GDPR).}}
\item[A] Discussion related to DevOps vs DevSecOps has been
  removed, as well as the early section on GDPR. Abbreviations have
  adopted the style recommended throughout. 

\item {\emph{There are repeated ``and'' in the sentence ``In particular the expectation of Privacy by design...'' Also, the ``by'' after ``and'' is not required.}}

{\emph{Please consider rephrasing to ``While both Privacy by design and privacy by defualt have been expected to be good practices''.}}

\item[A] These sentences have been rephrased and other typographical errors addressed.

\item {\emph{Please rephrase the following statement ``How might
  accreditation regard cybersecurity education, or help with it?'' as
  the sentence structure doesn't make much sense.}}

\item[A] This has been rephrased to ``{\emph{Can accreditation by
      Professional, Statutory and Regulatory Bodies enhance the
      provision of cybersecurity within a body's jurisdiction?}}'' The
  other research questions have also been rephrased to improve the
  clarity by which the intentions of the research are communicated.

\item {\emph{It is not clear what is the message the authors are
      conveying in the 2nd paragraph? It lists three different web
      links and a reference. Also, the web links in 2nd and 3rd
      paragraphs could be moved to the reference and they could be
      referenced with a number in this paragraph. Again, please use
      the acronym KA after introducing it as Knowledge Areas (KA) at
      the first instance of use.}}

\item[A] The authors apologise that these were note sections and should
  not have been present in the paper; these sections have been
  removed. The recommended approach to abbreviations has now been
  adopted throughout the paper.

\item  {\emph{What is JHD? Please expand?}}

\item[A] This has been removed this. 

\item {\emph{Please expand OWASP.}}

\item[A]Open Web Application Security Project has replaced the initial use of OWASP.

\item {\emph{The references to first author in the 2nd paragraph could
      be avoided and the sentence rephrased to highlight only the
      facts and not any personal experience.}}
 
\item[A] The references to the authors has been removed throughout and
  any potential conjecture removed.

\item {\emph{What does ``three accounted for the 36 of the 44'' mean?
      Please clarify. I am assuming the 36 and 44 refer to number of
      universities. Please add that clarification, if it is so.}}

\item[A] This did mean universities and has been updated to indicate
  this.

\item {\emph{``If there are only 82 instances of these fragments, how
      can 117 of them be verified?'' Please add clarification to this
      discussion.}}

\item[A] Typo on our part: this should have been 820, not 82.
\end{enumerate}

\section*{Referee 4}

\begin{enumerate}
\item ``{\emph{The problem is it is not a research paper in even the
      broadest interpretation of that sense. The RQs are invented
      based on topic areas the authors wish to riff on, only RQ3 comes
      (somewhat) close to a RQ with an associated presentation based
      on the literature and some analysis. The 1st 2 RQs, as well as
      the Introduction sections, are selectively argued,
      stream-of-consciousness almost writing. Entertaining,
      informative, arguable, debatable - but not research. No
      methodology is applied, no systematic review, etc. Instead a
      smattering of anecdotal information and quotes (and even more
      quotes in the footnotes) intended to convince the reader of a
      viewpoint that is never quite articulated clearly - until the
      end when it comes down to a small set of specific complaints
      about SQL Injection, XSS, and StackOverflow (which is overall a
      bit of a letdown as I felt as if this bloglike writing was
      building to a more impactful crescendo, some kind of
      kick-in-the-pants call to action for the community).}}''

\item[A] The paper has been revised to more clearly communicate the
  research approach employed, as well as the overall aims and
  intention of the paper (for example, new title, abstract and
  introductory framing). A research approach section has been
  added to further elicit the case study approach adopted. The
  research questions have been made more specific to more clearly
  communicate the intentions of the research. The writing style has
  been adjusted and a more formal approach taken. The conclusions have
  been extended and clarified.

\end{enumerate}

\section*{Programme Committee}

All addressed/no comments that require addressing in detail.

\end{document}
\item 
\item[A]
\item 
\item[A]
\item 
\item[A]


\end{document}
\bibliography{../../../../jhd}
\section{}
\begin{description}
\item
\end{description}
\begin{itemize}
\item
\end{itemize}
\begin{example}
\end{example}
\begin{definition}
\end{definition}
\begin{theorem}
\end{theorem}
\begin{enumerate}
\end{enumerate}
\begin{description}
\item[Theme 1]
\item[Theme 2]
\item[Theme 3]
\item[Theme 4]
\item[Theme 5]
\end{description}

