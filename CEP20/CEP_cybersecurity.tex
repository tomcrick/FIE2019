%%
%% This is file `sample-sigconf.tex',
%% generated with the docstrip utility.
%%
%% The original source files were:
%%
%% samples.dtx  (with options: `sigconf')
%% 
%% IMPORTANT NOTICE:
%% 
%% For the copyright see the source file.
%% 
%% Any modified versions of this file must be renamed
%% with new filenames distinct from sample-sigconf.tex.
%% 
%% For distribution of the original source see the terms
%% for copying and modification in the file samples.dtx.
%% 
%% This generated file may be distributed as long as the
%% original source files, as listed above, are part of the
%% same distribution. (The sources need not necessarily be
%% in the same archive or directory.)
%%
%% The first command in your LaTeX source must be the \documentclass command.
\documentclass[sigconf]{acmart}

%%
%% \BibTeX command to typeset BibTeX logo in the docs
\AtBeginDocument{%
  \providecommand\BibTeX{{%
    \normalfont B\kern-0.5em{\scshape i\kern-0.25em b}\kern-0.8em\TeX}}}

%% Rights management information.  This information is sent to you
%% when you complete the rights form.  These commands have SAMPLE
%% values in them; it is your responsibility as an author to replace
%% the commands and values with those provided to you when you
%% complete the rights form.
\setcopyright{acmcopyright}
\copyrightyear{2020}
\acmYear{2020}
\acmDOI{10.1145/1122445.1122456}

%% These commands are for a PROCEEDINGS abstract or paper.
\acmConference[CEP '20]{CEP '20: ACM Computing Education Practice}{January 9, 2020}{Durham, UK}
\acmBooktitle{CEP '20: Proceedings of the 3rd Conference on Computing Education Practice,
  June 9, 2020, Durham, UK}
\acmPrice{15.00}
\acmISBN{978-1-4503-9999-9/18/06}


%%
%% Submission ID.
%% Use this when submitting an article to a sponsored event. You'll
%% receive a unique submission ID from the organizers
%% of the event, and this ID should be used as the parameter to this command.
%%\acmSubmissionID{123-A56-BU3}

%%
%% The majority of ACM publications use numbered citations and
%% references.  The command \citestyle{authoryear} switches to the
%% "author year" style.
%%
%% If you are preparing content for an event
%% sponsored by ACM SIGGRAPH, you must use the "author year" style of
%% citations and references.
%% Uncommenting
%% the next command will enable that style.
%%\citestyle{acmauthoryear}

%%
%% end of the preamble, start of the body of the document source.
\begin{document}

%%
%% The "title" command has an optional parameter,
%% allowing the author to define a "short title" to be used in page headers.
\title{The challenges of teaching cybersecurity as part of mainstream computer science and related disciplines?}

%% 
%% The "author" command and its associated commands are used to define
%% the authors and their affiliations.
%% Of note is the shared affiliation of the first two authors, and the
%% "authornote" and "authornotemark" commands
%% used to denote shared contribution to the research.


%\begin{comment}

\author{Tom Crick}
\affiliation{%
  \institution{Swansea University}
  \city{Swansea}
  \country{UK}
}
\email{thomas.crick@swansea.ac.uk}


\author{James H. Davenport}
\affiliation{%
  \institution{ University of Bath}
  \city{Bath}
  \country{UK}
}
\email{j.h.davenport@bath.ac.uk}

\author{Alastair irons}
\affiliation{%
  \institution{ Sunderland University}
  \city{Sunderland}
  \country{UK}
}
\email{alastair.irons@sunderland.ac.uk}

\author{Tom Prickett}
\affiliation{%
  \institution{ Northumbria University}
  \city{Newcastle upon Tyne}
  \country{UK}
}
\email{tom.prickett@northumbria.ac.uk}
%\end{comment}


%%
%% By default, the full list of authors will be used in the page
%% headers. Often, this list is too long, and will overlap
%% other information printed in the page headers. This command allows
%% the author to define a more concise list
%% of authors' names for this purpose.
\renewcommand{\shortauthors}{Crick, Davenport,  Irons, and Prickett.}
%%
%% The abstract is a short summary of the work to be presented in the
%% article.
\begin{abstract}
  "Every Computer Science Degree Should Require a Course in Cybersecurity" **TO DO add reference**
  argued a recent article in the Harvard Business review. Computer Science and related degrees in the united Kingdom are evolving to meet this expectation. The BCS, The Charterted Institute for IT has been mandating this for a number of years \cite{Cricketal2019}. This may have encouraged these developments, although the discipline has been embedding cybersecurity before it was mandated by the BCS. Delivering cybersecurity effectively presents a number of challenges related to research and pedagogy. This article explores the progress to date, challenges for the future and highlights areas of potential  enhancement.
   
\end{abstract}

%%
%% The code below is generated by the tool at http://dl.acm.org/ccs.cfm
%% Please copy and paste the code instead of the example below.
%%
\begin{CCSXML}
<ccs2012>
<concept>
<concept_id>10002978</concept_id>
<concept_desc>Security and privacy</concept_desc>
<concept_significance>500</concept_significance>
</concept>
<concept>
<concept_id>10003456.10003457.10003527.10003529</concept_id>
<concept_desc>Social and professional topics~Accreditation</concept_desc>
<concept_significance>500</concept_significance>
</concept>
</ccs2012>
\end{CCSXML}

\ccsdesc[500]{Security and privacy}
\ccsdesc[500]{Social and professional topics~Accreditation}

% To do add cybersecurity to this

%% Keywords. The author(s) should pick words that accurately describe
%% the work being presented. Separate the keywords with commas.
\keywords{Accreditation}


%%
%% This command processes the author and affiliation and title
%% information and builds the first part of the formatted document.
\maketitle


\section {What is it?}
%A short description of the practice you're presenting

This paper explores the challenges related to research and pedagogy of delivering the this cybersecurity content.

\section {Why are you doing it?}	What happened before? What is it changing / replacing / improving? What gap is it filling?

Cybersecurity is becoming increasing pivotal to the operation of organisations of all sizes and all organisations are increasingly expected to make reasonable adjustments to protect their activities 

\begin{quote}
	``{\emph{...[need to] change the culture in your organisation around cyber security; to try to do for cyber what has been done so successfully for health and safety, for example, over the last ten years --- to get everybody to take it seriously; to take the risk management process seriously and drive that down through the organisation.}}''\\
	\hfill Robert Hannigan~\cite{Hannigan2019a}, former Director of GCHQ
\end{quote}


This focus on cybersecurity, includes calls for formal education -- school-level as well as tertiary -- to respond to this situation, at the individual level and via recommended curricula~\cite{mcgettrick-et-al:sigcse2014,ACM2017b} and professional accreditation requirements~\cite{BCS2018a,NCSC2017}. This is further reinforced by a wider focus on digital skills and computer science education reform, especially across the nations of the UK~\cite{brown-et-al:toce2014,murphy-et-al:programming2017,tryfonas+crick:petra2018}

Computer Science and related disciplines are evolving to meet these demands. However doing so is not without challenges. This paper explores the progress to date, challenges for the future and highlights a number of enhancement activities for the discipline.

\section {Where does it fit?	}A short description of your teaching context. You may, for instance, include a description of intake, class size, curriculum sequence; anything that's necessary for others to understand your situation. How do things work at your institution?

\section {Does it work?}	How do you know? Give some evidence of effectiveness in context.

The UK situation appears relatively advanced compared to other jurisdictions. 61\% of UK courses offer mandatory cybersecurity content, and this research was based on web scraping~\cite[Table 1]{Ruiz2019a}. As such it represents a lower bound since not all coverage will necessarily be clearly articulated in publicly available documentation online.

In many jurisdictions, Professional, Statutory and Regulatory Bodies (PSRBs) have responded to these expectations by adjusting their accreditation requirements. In the UK, BCS, The Chartered Institute for IT (BCS) has had a requirement to include information security in the curriculum since 2010, and has expected coverage of an agreed cybersecurity syllabus since 2015, with the result that all accredited universities should be compliant by 2020 (due to the five-year cycle). More precisely, accredited degrees have been expected to demonstrate coverage of ``{\emph{2.1.9 Knowledge and understanding of information security issues in relation to the design, development and the use of information systems}}'' \cite[p.~30]{BCS2018a} since 2010 with an enhanced cybersecurity related definition of what this entails since 2015 \cite[p.~17--18]{BCS2018a}.

It has been reported that this initiative is showing good progress \cite{Cricketal2019} with the majority university's already meeting these expectations and a further number evolving their provision so that they do so.

Collective challenges still persist related to underpinning resources, staffing and expected pedagogies.

\section {Who else has done this?}	Where did you get the idea from? (If from published reports, please include references). How did you find out about it? Was it easy/hard to adopt? What did you change?

In jurisdictions other than the UK, PSRB's have also adjusted their expectations to enhance the cybersecurity provision. For example in the United States of America (USA), the Association of Computing Machinery (ACM) has equally had cybersecurity (IAS: ``Information Assurance and Security'') in the curriculum since 2013~\cite{ACM2013a}, but it is not the accrediting body. The Accreditation Board for Engineering and Technology (ABET) is, and is requiring IAS with effect from the 2019-20 cycle (self-study reports due 1 July 2019): more precisely \cite[Table 3]{Oudshoornetal2018a} ``{\emph{The computing topics must include: \dots{} Principles and practices for secure computing\dots}}''. This should mean that  all accredited universities should be compliant by 2025 (due to their six-year cycle)

The ACM/IEEE-CS Joint Task Force on Computing Curricula~\cite[p.~97]{ACM2013a} takes a distinct view on the Knowledge Areas (KAs):

\begin{quote}
	``{\emph{The Information Assurance and Security KA is unique among the set of KAs presented here
			given the manner in which the topics are pervasive throughout other KAs.}}''
\end{quote}

It proposes nine ``core'' hours and 63.5 distributed across the other KAs. Nevertheless, the situation on the ground in the USA is different~\cite{Ackerman2019a}:

\begin{quote}
	``{\emph{Universities suffer shortcomings, as well. Roughly 85 of them offer undergraduate and/or graduate degrees in cybersecurity. There is a big catch, however. Far more diversified computer science programs, which attract substantially more students, don't mandate even one cybersecurity course.}}''
\end{quote}

However, whilst the accreditation regimes in the UK are mandating inclusion of cybersecurity there remain challenges in terms of the underpinning resources, the available Faculty resources and pedagogies required to effectively deliver the cybersecurity content.

\section {What will you do next?}	Will you vary this, or develop it further?
\section{Why are you telling us this?}	What is interesting, or useful, about this to someone else?

\section{Acknowledgments}
The authors wish to thank Sally Pearce, Academic Accreditation Manager at BCS, The Chartered Institute for IT for supplying the summary information related to accreditation of UK degree programmes. Many people, accreditors and accredited, have contributed to acreditation pratice in the UK (and elsewhere), and spreading good practice.  All authors' institutions are members of the Institute of Coding, an initiative funded by the Office for Students (England) and the Higher Education Funding Council for Wales.
%%
%% The next two lines define the bibliography style to be used, and
%% the bibliography file.
\bibliographystyle{ACM-Reference-Format}
%\bibliography{sample-base}
\bibliography{CEP2020}


\end{document}
\endinput
%%
%% End of file `sample-sigconf.tex'.
